\documentclass[aspectratio=169]{beamer}
\usetheme{metropolis}
\usepackage[utf8]{inputenc}
\usepackage[T1]{fontenc}
\usepackage{lmodern}
\usepackage{amsmath,amssymb,mathtools}
\usepackage{bm}
\usepackage{tikz}
\usepackage{booktabs}
\usepackage{graphicx}
\usepackage{hyperref}
\usepackage{caption}
\usepackage{minted} % compile with -shell-escape
\setminted{fontsize=\footnotesize, breaklines, autogobble, frame=lines}

\title{KAN Tutorial Slides}
\author{Daniel Precioso Garcelán}
\date{\today}
\begin{document}
\maketitle
\begin{frame}{Kolmogorov-Arnold Networks}

\end{frame}

\begin{frame}{Kolmogorov-Arnold Representation Theorem}
Let $ \Omega \subset \mathbb{R}^d $ be a bounded domain and let $ f: \Omega \rightarrow \mathbb{R} $ be a continuous function; i.e. $ f \in C(\Omega) $.

Then there exist continuous univariate functions
$$
\Phi_q: \mathbb{R} \to \mathbb{R}, \quad q = 1, \dots, 2d+1;
$$
and continuous univariate functions
$$
\phi_{pq}: \mathbb{R} \to \mathbb{R}, \quad p = 1, \dots, d; \quad q = 1, \dots, 2d+1;
$$
such that for every $\mathbf{x} = (x_1, \dots, x_d) \in \Omega$,
$$
f(\mathbf{x}) = \sum_{q=1}^{2d+1} \Phi_q \left( \sum_{p=1}^d \phi_{pq}(x_p) \right).
$$
\end{frame}

\begin{frame}{Kolmogorov-Arnold Representation Theorem}
For a bivariate function $f(x_1, x_2)$, we then have:

\begin{figure}
	\centering
	\resizebox{0.4\linewidth}{!}{\input{kan_diagram.tex}}
\end{figure}

Which is a simple KAN!
\end{frame}

\begin{frame}{B-Splines}
$\Phi_q$ and $\phi_{p,q}$ can be chosen from any family of continuous univariate functions. A common choice is the **B-spline** family.

A B-spline of degree $k$ is defined as:

$$B_k(x) = \sum_{i=0}^{n-k-1} P_i N_{i,k}(x)$$

where $n$ is the number of control points (length of the knot vector), and $P_i$ are the basis function weights.

And $N_{i,k}$ are the basis functions, following the standard **Cox-de Boor recursive definition**:

$$
N_{i,0}(x) =
\begin{cases}
1, & t_i \le x < t_{i+1} \\
0, & \text{otherwise}
\end{cases}
$$

$$
N_{i,k}(x) =
\frac{x - t_i}{t_{i+k} - t_i} N_{i,k-1}(x)
+
\frac{t_{i+k+1} - x}{t_{i+k+1} - t_{i+1}} N_{i+1,k-1}(x),
\quad k > 0
$$

where ${t_i} \in [t_1, t_n]$ is the **knot vector**, a non-decreasing sequence of real numbers.
\end{frame}

\begin{frame}{Grid Extension}

\end{frame}

\begin{frame}{Continuous Learning}
B-splines are made of basin functions that operate in small bounds of X. Learning new information in a part of X does not alter other regions of X. In contrary, traditional MLP risk **catastrohpic forgetting**.
\end{frame}

\begin{frame}{Symbolic Regression}

\end{frame}

\begin{frame}{Sparsibility}

\end{frame}

\end{document}
